\documentclass[10pt]{article}
 \usepackage[margin=1in]{geometry} 
\usepackage{amsmath,amsthm,amssymb,amsfonts,color, titling}
 \usepackage{listings}

\setlength{\droptitle}{-20mm} 

\usepackage[colorlinks=true,linkcolor=red,urlcolor=blue]{hyperref}

\title{In-class assignment \# 4}
\author{Brian O'Shea, \\PHY-905-003, Computational Astrophysics and
  Astrostatistics\\Spring 2017}
 \date{} % leave blank to have no date

\begin{document}
 
\maketitle

\vspace{-5mm}

\noindent \textbf{Instructions:}   We're going to use the code for
your pre-class assignment, where you modeled the behavior of a mass on
a spring over some interval of time.  You've already implemented Euler
method and a simple predictor-corrector method.  Now, implement (1)
either the
  \href{https://en.wikipedia.org/wiki/Semi-implicit_Euler_method}{Euler-Cromer
  method}  or the 
  \href{https://en.wikipedia.org/wiki/Midpoint_method}{midpoint
    method} and (2) the
  \href{https://en.wikipedia.org/wiki/Runge%E2%80%93Kutta_methods}{4th
                                order Runge-Kutta method}, and answer
                              the following questions:

\begin{itemize}

\item For given timestep sizes, $\Delta t = 0.1 \pi, 0.01 \pi,$ and
  $0.001 \pi$, what is the
  difference in the relative change in energy between the Euler
  methods and these two methods?

\item Assume you wish to maintain energy to a given level of accuracy
  - say 0.01\% between $t=0$ and $t=4\pi$.  How many time steps of
  each of your methods do you need to reach that level of accuracy?
  How many total floating-point operations is that for each method for
  the entire integration?

\end{itemize}

\noindent
Make some notes in \texttt{ANSWERS.md}, and
we'll also discuss it in class.

\vspace{5mm}

\noindent 
\textbf{What to turn in:} Turn in \texttt{ANSWERS.md}, any
source code you wrote, any plots you created (and the scripts you used
to create them).  \textbf{Do not} turn in object files or
executables!

\end{document}